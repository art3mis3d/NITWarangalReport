\section{Introduction}
Alzheimer's disease is a debilitating neurodegenerative disorder that \\poses significant challenges for early detection and treatment\cite{smith2020}. 
As the most prevalent form of dementia, 
Alzheimer's disease exerts a substantial burden on individuals, 
families, 
and society at large\cite{alzassociation2019}. 
It manifests as a progressive decline in cognitive function, 
memory impairment, 
and compromised daily functioning, 
ultimately resulting in a loss of autonomy and diminished quality of life\cite{mckhann2011}.\\

Early detection of Alzheimer's disease is of paramount importance due to several compelling reasons. 
Firstly, 
it enables the timely implementation of intervention and management strategies, 
potentially decelerating disease progression and ameliorating symptoms\cite{livingston2020}. 
Secondly, 
early detection empowers individuals and their families to proactively plan for the future, 
make informed decisions, 
and access pertinent support and resources\cite{karlawish2011}.\\

Conventional diagnostic modalities for Alzheimer's disease, 
such as neuroimaging techniques and cognitive assessments, 
possess inherent limitations in terms of cost, 
accessibility, 
and sensitivity, 
particularly during the initial stages of the disease\cite{dubois2014}. 
Consequently, 
alternative diagnostic approaches, 
including the utilization of biosensors, 
have garnered substantial interest\cite{nguyen2017}.\\

Biosensors represent cutting-edge devices that amalgamate biological components, 
such as antibodies or enzymes, 
with transducers, 
enabling the precise detection and quantification of specific biomarkers or analytes\cite{turner1987}. 
These innovative devices have demonstrated tremendous potential across various domains, 
including healthcare, 
by facilitating rapid, 
sensitive, 
and specific detection of target molecules\cite{lange2008}.\\

Which accumulates in the brain and gives rise to the characteristic plaques indicative of the disease\cite{hardy1992}. 
Biosensors can be tailored to identify and quantify A$\beta$
levels in biological samples, 
including cerebrospinal fluid or blood\cite{soares2019}. 
By capitalizing on the remarkable specificity and sensitivity of biosensors, 
researchers endeavor to develop non-invasive and cost-effective diagnostic tools for the early detection of Alzheimer's disease\cite{manfredsson2020}.\\

Employing biosensors for the early diagnosis of Alzheimer's disease confers several advantages. 
It enables the detection of biomarkers in a minimally invasive manner, 
facilitating convenient and frequent monitoring\cite{wang2005}. 
Furthermore, 
biosensors have the capacity to detect subtle fluctuations in biomarker levels, 
even preceding the manifestation of clinical symptoms\cite{batool2019}. 
Moreover, 
these devices can be seamlessly integrated into point-of-care settings, 
thereby promoting widespread screening and augmenting patient access to early diagnostic services\cite{ding2018}.\\

In summary, 
early detection of Alzheimer's disease is indispensable for efficacious intervention and management. 
Biosensors hold immense promise as groundbreaking tools for the early diagnosis of Alzheimer's disease, 
offering the potential for sensitive, 
specific, 
and non-invasive detection of biomarkers, 
notably amyloid beta\cite{janson2004}. 
Through their development and utilization, 
biosensors possess the capacity to revolutionize early detection strategies, 
empowering timely interventions, 
and enhancing the well-being of individuals afflicted by this devastating disorder\cite{kumar2018}.\\
