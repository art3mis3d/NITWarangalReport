Alzheimer's disease is a neurodegenerative disorder characterized by cognitive decline and memory impairment. 
Early detection of Alzheimer's disease is crucial for effective intervention and treatment. 
Biosensors have emerged as promising tools for the early detection of Alzheimer's disease due to their sensitivity, 
specificity, 
and potential for non-invasive detection. 
In this article, 
I present a comprehensive review of computational approaches used in the design of biosensors for the early detection of Alzheimer's disease. 
We discuss various strategies employed in biosensor design, 
including the selection of appropriate biomarkers, 
molecular docking simulations, 
and machine learning algorithms. 
Furthermore, 
I highlight recent advancements in the field, 
such as the integration of nanomaterials and microfluidics in biosensor platforms. 
By leveraging computational approaches, 
researchers can optimize biosensor performance, 
enhance detection sensitivity, 
and improve the accuracy of early Alzheimer's disease diagnosis. 
This preprint provides valuable insights into the current state of computational approaches for biosensor design and their potential impact on early detection strategies for Alzheimer's disease.